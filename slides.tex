\documentclass[aspectratio=169]{beamer}
\usepackage[utf8]{inputenc}
\usepackage[T1]{fontenc}
\usepackage{tikz}
\usetheme[showmaxslides, darkmode, nofooter]{pureminimalistic}

\usepackage{pgfpages}
\setbeameroption{show notes on second screen}
% \pgfpagesuselayout{2 on 1}[a4paper,border shrink=10mm]

\usepackage[backend=biber, doi=false, maxbibnames=2, maxcitenames=2,style=numeric, sorting=none, url=false, eprint=false]{biblatex}
\addbibresource{demo_bib.bib}

% this makes it possible to add backup slides, without counting them
\usepackage{appendixnumberbeamer}
\renewcommand{\appendixname}{\texorpdfstring{\translate{appendix}}{appendix}}
\renewcommand{\logoheader}{}

% if loaded after begin{document} a warning will appear: "pdfauthor already used"
\title[Debugging RxJS-based Applications]{Getting Rid of the \texttt{console.log}:\\Debugging RxJS-based Applications}
\author{Manuel Alabor}
\institute{Eastern Switzerland University of Applied Sciences}
\date{\today}


\begin{document}
  \begin{frame}{Default Options}
    Lorem ipsum dolor sit amet, consetetur sadipscing elitr,
    sed diam nonumy eirmod tempor invidunt ut labore et dolore
    magna aliquyam erat, sed diam voluptua.
    At vero eos et accusam et justo duo dolores et ea rebum.
    Stet clita kasd gubergren, no sea takimata sanctus est
    Lorem ipsum dolor sit amet.
  \end{frame}
\end{document}

% \documentclass[aspectratio=169]{beamer}
% % should also look nice for the classic aspectratio
% % of course, than the text has to be refitted
% % \documentclass{beamer}
% \usepackage[utf8]{inputenc}
% \usepackage[T1]{fontenc}
% \usepackage{tikz}
% \usetheme[darkmode, showmaxslides]{pureminimalistic}
% % \renewcommand{\pageword}{}
% % \renewcommand{\logoheader}{\vspace{1.5em}}
% \usepackage[backend=biber, doi=false, maxbibnames=2, maxcitenames=2,%
%             style=numeric, sorting=none, url=false, eprint=false]{biblatex}
% \addbibresource{bibliography.bib}
% % this makes it possible to add backup slides, without counting them
% \usepackage{appendixnumberbeamer}
% \renewcommand{\appendixname}{\texorpdfstring{\translate{appendix}}{appendix}}


% \title[short title]{This is the normal length of a research paper:
% always longer than you would expect}
% \author{Kai Norman Clasen}
% \institute{Insitute name}
% \date{\today}

% % \documentclass{beamer}
% % \documentclass[handout]{beamer}
% % \usepackage{pgfpages}
% % \setbeameroption{show notes on second screen}
% % \pgfpagesuselayout{2 on 1}[a4paper,border shrink=10mm]

% % \usepackage{ccicons}
% % \usepackage{fontawesome}
% % \usepackage{hyperref}

% % \usetheme{focus}
% % \usetheme{pureminimalistic}
% % \titlegraphic{\includegraphics[width=\textwidth/3]{./figures/brittney-weng-LycoZLp4jh0-unsplash}}
% % \title{Getting rid of the \texttt{console.log}}
% % \subtitle{Debugging RxJS-based Applications\\}
% % \author{\vspace{0.5cm}Manuel Alabor}
% % \institute{\small{Eastern Switzerland University\\ of Applied Sciences\\Rapperswil, Switzerland}}
% % \date{\vspace{0.9cm}\small{16 November 2020}}

% \begin{document}

% \section*{Welcome}

% \maketitle

% % \note[itemize]{
% % 	\item Welcome
% % 	\item Debugging of RxJS-based Applications
% % }

% % \begin{frame}{Who Am I?}
% % 	\begin{itemize}
% % 		\item Foobar\bigskip
% % 		\item Foobar\bigskip
% % 		\item Foobar\bigskip
% % 		\item Foobar\bigskip
% % 	\end{itemize}
% % \end{frame}

% % % \begin{frame}[focus]
% % % 	\includegraphics[width=\textwidth/3]{figures/rxjs-logo.png}
% % % 	\\
% % % 	\vspace{0.5cm}
% % % 	\Huge{RxJS}
% % % \end{frame}

% % \note[itemize] {
% % 	\item
% % }

% % \section*{Introduction}

% % \begin{frame}{}

% % \end{frame}

% % \note[itemize] {
% % 	\item
% % }


% % \begin{frame}{}

% % \end{frame}

% % \note[itemize] {
% % 	\item
% % }

% % \section{Interviews and War Stories}

% % \begin{frame}{Interviews}

% % \end{frame}

% % \note[itemize] {
% % 	\item
% % }


% % \begin{frame}{War Stories}

% % \end{frame}

% % \note[itemize] {
% % 	\item
% % }


% % \begin{frame}{Results}

% % \end{frame}

% % \note[itemize] {
% % 	\item
% % }

% % \section{Observational Study}
% % \begin{frame}{Hypothesis}

% % \end{frame}

% % \note[itemize] {
% % 	\item
% % }


% % \begin{frame}{Design}

% % \end{frame}

% % \note[itemize] {
% % 	\item
% % }


% % \begin{frame}{Problem Example}

% % \end{frame}

% % \note[itemize] {
% % 	\item
% % }


% % \begin{frame}{Results}

% % \end{frame}

% % \note[itemize] {
% % 	\item
% % }

% % \section{Conclusion}
% % \begin{frame}{}

% % \end{frame}

% % \note[itemize] {
% % 	\item
% % }


% % \begin{frame}{Future Work}

% % \end{frame}

% % \note[itemize] {
% % 	\item
% % }



% % % \note[itemize]{
% % % 	\item This was my first seminar paper done
% % % 	\item My goals:
% % % 	\item \textbf{Learn} scientific work style
% % % 	\item \textbf{Review} report
% % % 	\item \textbf{Transfer} a topic to frontend engineering
% % % 	\item \textbf{Write and present} my own research report\bigskip
% % % 	\item To do so, Markus came up with an interesting paper:
% % % }

% % % \begin{frame}[focus]
% % % 	\emph{``On the Positive Effect of Reactive Programming on Software Comprehension: An Empirical Study''}
% % % 	\\\bigskip
% % % 	\small{Salvaneschi et al. \cite{7827078}}
% % % \end{frame}

% % % \note[itemize]{
% % % 	\item On the positive effect of reactive programming on software comprehension
% % % 	\item An empirical study
% % % 	\item Research Group around Guido Salvaneschi
% % % 	\item 2015\bigskip
% % % }

% % % \subsection*{Research Questions}

% % % \begin{frame}{RQ1: Design Sciences and \\Empirical Software Engineering}
% % % 	\begin{enumerate}
% % % 		\item[RQ1.1] Which \textbf{empirical research methods}, approaches and concepts \textbf{were applied} by Salvaneschi et al. \cite{7827078}?\bigskip
% % % 		\item[RQ1.2] Are these research methods, approaches and concepts \textbf{applied well}, what could have been done better?\bigskip
% % % 		\item[RQ1.3] \textbf{Do} Salvaneschi et al. \cite{7827078} \textbf{meet FAIR principles} \cite{2019arXiv190805986H} \cite{wilkinson:2016} with their work?

% % % 	\end{enumerate}
% % % \end{frame}

% % % \note[itemize]{
% % % 	\item \textbf{Answer} 7 research questions
% % % 	\item Grouped in \textbf{2 sets} of research questions
% % % 	\item First: Research Principles
% % % 	\item Which \textbf{empirical research methods}?
% % % 	\item Are they \textbf{applied well}?
% % % 	\item Do they meet \textbf{FAIR} principles?
% % % }

% % % \section{Review of Salvaneschi et al.}
% % % \note[itemize]{
% % % 	\item In order to review Salvaneshi et al.:
% % % 	\item Present you two score cards
% % % 	\item One card per question set
% % % }

% % % \subsection*{Research Principle Score Card}
% % % \begin{frame}{Research Principle Score Card}
% % % 	\begin{exampleblock}{\faicon{thumbs-o-up} Empirical Software Engineering}
% % % 		\begin{itemize}
% % % 			\item \textbf{Careful} study \textbf{setup} and \textbf{execution}
% % % 			\item \textbf{Transparent} description of \textbf{threats to validity}
% % % 		\end{itemize}
% % % 	\end{exampleblock}

% % % 	\begin{alertblock}{\faicon{thumbs-o-down} Design Sciences}
% % % 		\textbf{No evidence} for the \textbf{application} of \textbf{design sciences} found
% % % 	\end{alertblock}

% % % 	\begin{alertblock}{\faicon{thumbs-o-down} FAIR Research Principles}
% % % 		Lack of \textbf{complementary information} violates \textbf{Reusability} principle
% % % 	\end{alertblock}
% % % \end{frame}

% % % \note[itemize]{
% % % 	\item \textbf{Empirical Software Engineering}: Well done, nothing to complain\bigskip
% % % 	\item \textbf{Design Sciences}: No clear evidence for use
% % % 	\item This is one study of many... Did they use Design Sciences for their high level planning?\bigskip
% % % 	\item \textbf{FAIR}: Good, except: Complementary information is missing
% % % 	\item Survey Tool and full code samples would improve value significantly
% % % }


% % % \subsection*{Synthetic Application}
% % % \begin{frame}{Synthetic Application: Scenario}
% % % 	\begin{columns}
% % % 		\column{0.6\textwidth}
% % % 			\begin{itemize}
% % % 				\item \textbf{Fetch} and \textbf{combine remote} profile and avatar \textbf{resource}\bigskip
% % % 				\item Avatar \textbf{depends} on profile resource\bigskip
% % % 			\end{itemize}

% % % 		\column{0.4\textwidth}
% % % 			\begin{block}{S1: \textbf{Modelling Data Flow}}
% % % 				\textbf{Fetch} a users \textbf{complete} profile \textbf{information} (name as well as their avatar picture) from \textbf{two} distinct \textbf{data sources}. The \textbf{profile} information \textbf{contains} required input to fetch the \textbf{avatar}. \textbf{Show} the combined information in the user interface \textbf{at once}.
% % % 			\end{block}
% % % 	\end{columns}
% % % \end{frame}

% % % \note[itemize]{
% % % 	\item Synthetic application: Model data flow
% % % 	\item Scenario: Show a users profile information with avatar
% % % 	\item Fetch an combine two remote data sources
% % % 	\item To fetch avatar, profile is required to be present\bigskip
% % % 	\item With reactive programming, this looks like:
% % % }

% % % \begin{frame}[focus]
% % % 	Q \& A
% % % \end{frame}

% % \appendix
% % \begin{frame}[allowframebreaks]{References}
% % 	\bibliographystyle{plain}
% % 	\bibliography{bibliography}
% % \end{frame}

% \end{document}